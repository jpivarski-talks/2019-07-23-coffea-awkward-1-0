\pdfminorversion=4
\documentclass[aspectratio=169]{beamer}

\mode<presentation>
{
  \usetheme{default}
  \usecolortheme{default}
  \usefonttheme{default}
  \setbeamertemplate{navigation symbols}{}
  \setbeamertemplate{caption}[numbered]
  \setbeamertemplate{footline}[frame number]  % or "page number"
  \setbeamercolor{frametitle}{fg=white}
  \setbeamercolor{footline}{fg=black}
}

\usepackage[english]{babel}
\usepackage[utf8x]{inputenc}
\usepackage{tikz}
\usepackage{courier}
\usepackage{array}
\usepackage{bold-extra}
\usepackage{minted}
\usepackage[thicklines]{cancel}
\usepackage{fancyvrb}

\xdefinecolor{dianablue}{rgb}{0.18,0.24,0.31}
\xdefinecolor{darkblue}{rgb}{0.1,0.1,0.7}
\xdefinecolor{darkgreen}{rgb}{0,0.5,0}
\xdefinecolor{darkgrey}{rgb}{0.35,0.35,0.35}
\xdefinecolor{darkorange}{rgb}{0.8,0.5,0}
\xdefinecolor{darkred}{rgb}{0.7,0,0}
\definecolor{darkgreen}{rgb}{0,0.6,0}
\definecolor{mauve}{rgb}{0.58,0,0.82}

\title[2019-07-23-coffea-awkward-1-0]{Future of Awkward-Array}
\author{Jim Pivarski}
\institute{Princeton University -- IRIS-HEP}
\date{August 23, 2019}

\usetikzlibrary{shapes.callouts}

\begin{document}

\logo{\pgfputat{\pgfxy(0.11, 7.4)}{\pgfbox[right,base]{\tikz{\filldraw[fill=dianablue, draw=none] (0 cm, 0 cm) rectangle (50 cm, 1 cm);}\mbox{\hspace{-8 cm}\includegraphics[height=1 cm]{princeton-logo-long.png}\hspace{0.1 cm}\raisebox{0.1 cm}{\includegraphics[height=0.8 cm]{iris-hep-logo-long.png}}\hspace{0.1 cm}}}}}

\begin{frame}
  \titlepage
\end{frame}

\logo{\pgfputat{\pgfxy(0.11, 7.4)}{\pgfbox[right,base]{\tikz{\filldraw[fill=dianablue, draw=none] (0 cm, 0 cm) rectangle (50 cm, 1 cm);}\mbox{\hspace{-8 cm}\includegraphics[height=1 cm]{princeton-logo.png}\hspace{0.1 cm}\raisebox{0.1 cm}{\includegraphics[height=0.8 cm]{iris-hep-logo.png}}\hspace{0.1 cm}}}}}

% Uncomment these lines for an automatically generated outline.
%\begin{frame}{Outline}
%  \tableofcontents
%\end{frame}

% START START START START START START START START START START START START START

\begin{frame}[fragile]{First of all: combinatorics language}
\vspace{0.3 cm}
\textcolor{darkblue}{Benchmark 8}: ``For events with at least three leptons and a same-flavor opposite-sign lepton pair, find the same-flavor opposite-sign lepton pair with the mass closest to 91.2 GeV and plot the pT of the leading other lepton.''

\small
\vspace{0.2 cm}
\begin{Verbatim}[commandchars=\\\{\}]
leptons = electrons \textcolor{blue}{union} muons

\textcolor{blue}{cut} count(leptons) >= \textcolor{mauve}{3} \textcolor{blue}{named} "three_leptons" \{
    Z = electrons \textcolor{blue}{as} (lep1, lep2) \textcolor{blue}{union} muons \textcolor{blue}{as} (lep1, lep2)
            \textcolor{blue}{where} lep1.charge != lep2.charge
            \textcolor{blue}{min by} abs(mass(lep1, lep2) - \textcolor{mauve}{91.2})

    third = leptons \textcolor{blue}{except} [Z.lep1, Z.lep2] \textcolor{blue}{max by} pt

    \textcolor{blue}{hist} third.pt \textcolor{blue}{by} regular(\textcolor{mauve}{100}, \textcolor{mauve}{0}, \textcolor{mauve}{250})
        \textcolor{blue}{named} \textcolor{mauve}{"third_pt"} \textcolor{blue}{titled} \textcolor{mauve}{"Leading other lepton pT"}
\}
\end{Verbatim}

\mbox{ } \hfill \textcolor{blue}{\underline{\url{https://github.com/jpivarski/PartiQL}}} \hfill \mbox{ }
\end{frame}

\end{document}
