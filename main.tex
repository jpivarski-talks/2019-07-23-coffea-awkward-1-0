\pdfminorversion=4
\documentclass[aspectratio=169]{beamer}

\mode<presentation>
{
  \usetheme{default}
  \usecolortheme{default}
  \usefonttheme{default}
  \setbeamertemplate{navigation symbols}{}
  \setbeamertemplate{caption}[numbered]
  \setbeamertemplate{footline}[frame number]  % or "page number"
  \setbeamercolor{frametitle}{fg=white}
  \setbeamercolor{footline}{fg=black}
}

\usepackage[english]{babel}
\usepackage[utf8x]{inputenc}
\usepackage{tikz}
\usepackage{courier}
\usepackage{array}
\usepackage{bold-extra}
\usepackage{minted}
\usepackage[thicklines]{cancel}
\usepackage{fancyvrb}

\xdefinecolor{dianablue}{rgb}{0.18,0.24,0.31}
\xdefinecolor{darkblue}{rgb}{0.1,0.1,0.7}
\xdefinecolor{darkgreen}{rgb}{0,0.5,0}
\xdefinecolor{darkgrey}{rgb}{0.35,0.35,0.35}
\xdefinecolor{darkorange}{rgb}{0.8,0.5,0}
\xdefinecolor{darkred}{rgb}{0.7,0,0}
\xdefinecolor{mauve}{rgb}{0.58,0,0.82}
\xdefinecolor{brown}{rgb}{0.64,0.16,0.16}
\xdefinecolor{reallydarkgreen}{rgb}{0,0.25,0}

\title[2019-07-23-coffea-awkward-1-0]{Future of Awkward-Array}
\author{Jim Pivarski}
\institute{Princeton University -- IRIS-HEP}
\date{August 23, 2019}

\usetikzlibrary{shapes.callouts}

\begin{document}

\logo{\pgfputat{\pgfxy(0.11, 7.4)}{\pgfbox[right,base]{\tikz{\filldraw[fill=dianablue, draw=none] (0 cm, 0 cm) rectangle (50 cm, 1 cm);}\mbox{\hspace{-8 cm}\includegraphics[height=1 cm]{princeton-logo-long.png}\hspace{0.1 cm}\raisebox{0.1 cm}{\includegraphics[height=0.8 cm]{iris-hep-logo-long.png}}\hspace{0.1 cm}}}}}

\begin{frame}
  \titlepage
\end{frame}

\logo{\pgfputat{\pgfxy(0.11, 7.4)}{\pgfbox[right,base]{\tikz{\filldraw[fill=dianablue, draw=none] (0 cm, 0 cm) rectangle (50 cm, 1 cm);}\mbox{\hspace{-8 cm}\includegraphics[height=1 cm]{princeton-logo.png}\hspace{0.1 cm}\raisebox{0.1 cm}{\includegraphics[height=0.8 cm]{iris-hep-logo.png}}\hspace{0.1 cm}}}}}

% Uncomment these lines for an automatically generated outline.
%\begin{frame}{Outline}
%  \tableofcontents
%\end{frame}

% START START START START START START START START START START START START START

\begin{frame}[fragile]{First of all: combinatorics language}
\vspace{0.3 cm}
\textcolor{darkblue}{Benchmark 8}: ``For events with at least three leptons and a same-flavor opposite-sign lepton pair, find the same-flavor opposite-sign lepton pair with the mass closest to 91.2 GeV and plot the pT of the leading other lepton.''

\small
\vspace{0.2 cm}
\begin{Verbatim}[commandchars=\\\{\}]
leptons = electrons \textcolor{darkgreen}{\textbf{union}} muons

\textcolor{darkgreen}{\textbf{cut}} \textcolor{darkblue}{count}(leptons) >= \textcolor{brown}{3} \textcolor{darkgreen}{\textbf{named}} \textcolor{brown}{"three_leptons"} \{
    Z = electrons \textcolor{darkgreen}{\textbf{as}} (lep1, lep2) \textcolor{darkgreen}{\textbf{union}} muons \textcolor{darkgreen}{\textbf{as}} (lep1, lep2)
            \textcolor{darkgreen}{\textbf{where}} lep1\textcolor{darkblue}{.charge} != lep2\textcolor{darkblue}{.charge}
            \textcolor{darkgreen}{\textbf{min by}} \textcolor{darkblue}{abs}(\textcolor{darkblue}{mass}(lep1, lep2) - \textcolor{brown}{91.2})

    third = leptons \textcolor{darkgreen}{\textbf{except}} [Z\textcolor{darkblue}{.lep1}, Z\textcolor{darkblue}{.lep2}] \textcolor{darkgreen}{\textbf{max by}} pt

    \textcolor{darkgreen}{\textbf{hist}} third\textcolor{darkblue}{.pt} \textcolor{darkgreen}{\textbf{by}} regular(\textcolor{brown}{100}, \textcolor{brown}{0}, \textcolor{brown}{250})
        \textcolor{darkgreen}{\textbf{named}} \textcolor{brown}{"third_pt"} \textcolor{darkgreen}{\textbf{titled}} \textcolor{brown}{"Leading other lepton pT"}
\}
\end{Verbatim}

\vspace{-0.1 cm}
\mbox{ } \hfill \textcolor{blue}{\underline{\url{https://github.com/jpivarski/PartiQL}}} \hfill \mbox{ }
\end{frame}

\begin{frame}{The ``key'' point}
\vspace{-0.25 cm}
\begin{columns}
\column{0.4\linewidth}
\begin{center}
muons

\vspace{0.1 cm}
\begin{tabular}{| c | c c c c |}
\hline id & $p_T$ & $\eta$ & $\phi$ & \ldots \\\hline
\#0(0, 0) & \ldots & \ldots & \ldots & \ldots \\
\#0(0, 1) & \ldots & \ldots & \ldots & \ldots \\
\#0(0, 2) & \ldots & \ldots & \ldots & \ldots \\
\#0(2, 0) & \ldots & \ldots & \ldots & \ldots \\
\#0(2, 1) & \ldots & \ldots & \ldots & \ldots \\\hline
\end{tabular}
\end{center}

\column{0.4\linewidth}
\begin{center}
jets

\vspace{0.1 cm}
\begin{tabular}{| c | c c c c |}
\hline id & $p_T$ & $\eta$ & $\phi$ & \ldots \\\hline
\#1(0, 0) & \ldots & \ldots & \ldots & \ldots \\
\#1(0, 1) & \ldots & \ldots & \ldots & \ldots \\
\#1(1, 0) & \ldots & \ldots & \ldots & \ldots \\
\#1(2, 0) & \ldots & \ldots & \ldots & \ldots \\
\#1(2, 1) & \ldots & \ldots & \ldots & \ldots \\
\#1(2, 2) & \ldots & \ldots & \ldots & \ldots \\
\#1(2, 3) & \ldots & \ldots & \ldots & \ldots \\
\#1(2, 4) & \ldots & \ldots & \ldots & \ldots \\
\hline
\end{tabular}
\end{center}
\end{columns}

\vspace{0.25 cm}
Leaves of an awkward-array structure are assigned unique identifiers (``surrogate keys'' in database terminology). Identity follows particles through mathematical operations, so that filtered (\textcolor{darkgreen}{\textbf{where}}) and recombined (\textcolor{darkgreen}{\textbf{join}}, \textcolor{darkgreen}{\textbf{union}}) particle attributes match up.

\vspace{0.25 cm}
Jagged arrays with different key indexes (e.g. {\tt \#0} and {\tt \#1}) can't be combined, even if they accidentally have the right number per event.
\end{frame}

\begin{frame}{Ultimate goal: multi-paradigm awkward-arrays}
\large
\vspace{0.5 cm}
{\Large Want to use the same data structure with}

\vspace{0.15 cm}
\begin{itemize}
\item Numpy-style array-at-a-time operations in Python (``classic awkward''),
\item declarative (order-independent) query language with surrogate keys, and
\item procedural programming (traditional for-loops) in Numba,
\end{itemize}

\vspace{0.15 cm}
so that an analysis can start with a few lines of array-at-a-time work, enter a declarative block to do some complex combinatorics, and then a Numbafied function for something that can't be expressed well in either abstraction.
\end{frame}

\begin{frame}{Awkward 1.0}
\large
\vspace{0.5 cm}
{\Large Consolidation of {\tt awkward}, {\tt awkward-numba}, {\tt awkward-cpp}, etc.\ into a single package that shares code}

\begin{enumerate}
\item because it would simplify many implementations, allowing for greater generality (e.g.\ {\normalsize \mintinline{python}{JaggedArray.__getitem__}}) and eliminating Numpy corner-case bugs (e.g.\ max of empty array).

\end{enumerate}

\end{frame}

\begin{frame}{}
\end{frame}

\end{document}
